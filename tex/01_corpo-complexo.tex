% Seção 01: O Corpo Complexo ===============================================

\section{O Corpo dos Números Complexos}
\label{sec:corpo}

Definir o Conjunto dos Números Complexos, inicialmente, por 
\[
  \mathbb{C} = \left\{\, a + bi; \; a,b \in \mathbb{R} \text{ e } i^2 = -1 \,\right\},
\]
pode esconter algumas coisas fundamentais no entendimento desse novo conjunto. 
Afinal\ldots\ 
Seria essa soma entre $a$ e $bi$ é usual?
O que significa $i^2 = -1$?
Como você explicaria o seguinte Paradoxo?

\begin{paradoxo}
  Considerando, $ i^2 = -1$, temos:
  \[
    -1 = i^2 = i \cdot i = \sqrt{-1}\cdot \sqrt{-1} = \sqrt{(-1)(-1)} = \sqrt{1} = 1
  \]
Portanto, $-1 = 1$?
\end{paradoxo}

Mas, como Gauss já dizia: \textit{Na Matemática não existem controvérsias verdadeitas}! 

Faz-se necessário uma formalização na definição dos números complexos. 

\subsection{Formalizando as coisas} %---------------------------------------

Precisamos entender os Númetos Complexos não como uma ``sacola de números'', mas 
como uma ``corpo de números'', ou seja, um conjunto com uma estrutura algébrica 
associada, preferencialmente já conhecida.

Hamilton, matemático, físico e astrónomo irlandês, foi que trouxe uma definição 
adequada aos números complexos, sendo adotada até os dias atuais. 
Nela vê-se o conjunto dos números complexos como um conjunto de pares ordenados 
munidos com as operações de soma e produdo, este último bem peculiar, mas que 
oferece subsídios para verificar que essa estrutura é um Corpo. 


\begin{proposicao}
  Considere o subconjunto $\mathcal{C} \subset \mathbb{R}^2$, não vazio, munido 
  das seguintes operações de \textsf{soma} e \textsf{produdo}, dadas por 
  \[
    \begin{array}{rcc}
      + \colon \mathcal{C}\times \mathcal{C} & \to & \mathcal{C} \\
       (x, y) & \mapsto & x + y 
    \end{array}
    \quad 
    \text{ e }
    \quad 
    \begin{array}{rcc}
      \cdot \colon \mathcal{C}\times \mathcal{C} & \to & \mathcal{C} \\
       (x, y) & \mapsto & x \cdot y 
    \end{array}
  \]
  Então, são satisfeitas as seguintes condições, para $z,w,t \in \mathbb{C}$:

  \begin{enumerate}
      \item[(i)] $z + (w + t) = (z + w) + t$
      \item[(ii)] $z + w = w + z$
      \item[(iii)] $ \exists\ 0 \in \mathcal{C}$, tal que $z + 0 = z$
      \item[(iv)] $ \exists\ -z \in \mathcal{C}$, tal que $ z + (-z) = 0 $
      \item[(v)] $z(wt) = (zw)t$
      \item[(vi)] $zw = wz$
      \item[(vii)] $ \exists\ 1 \in \mathcal{C} $, tal que $z \cdot 1 = z$, $\forall z$
      \item[(viii)] $ \exists\ z^{-1} \in \mathcal{C}$, tal que $z\cdot z^{-1} = 1$, para todo $z \neq 0$
      \item[(ix)] $ z (w + t) = zw + zt  $
  \end{enumerate}  
\end{proposicao}

\begin{definicao}[Corpo dos Números Complexos]
  A esse conjunto $\mathcal{C}$, definido acima, denominamos \textsf{Corpo dos Números Complexos} e 
  será denotado por $\mathbb{C}$.
\end{definicao}

\begin{obs}
  É possível mostrar algumas identificaçãos, tais como 
  \begin{enumerate}
    \item $ (x, 0) \sim x $
    \item $ (0, 1)^2 = -1 $
    \item $ (0, y) \sim yi $
  \end{enumerate}
  Assim, um número complexo $ z = (x, y) $, pode ser identificado por 
  \begin{align}
    z &= (x, y) \notag\\
      &= (x, 0) + (0, y) \notag \\
      &= x + yi \label{eq:algebrica}
  \end{align}
\end{obs}

Dizemos que \eqref{eq:algebrica} está na \textbf{Forma Algébrica}. 

\subsection{Representação Matricial} %--------------------------------------

Também é possível representar um número complexo $ z = a + bi $, na forma 
\textbf{Matricial}: 
\[
  z = 
  \begin{bmatrix}
    a & -b \\
    b &  a
  \end{bmatrix}
\]

