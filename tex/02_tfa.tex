% Sec 2: TFA 

\section{Teorema Fundamental da Álgebra}
\label{sec:TFA}

Na Seção~\ref{sec:corpo}, vimos qualquer coisa sobre números complexos. 
Então, nessa seção, demonstraremos o \textsf{Teorema Fundamental da Álgebra} (TFA). 
Nunca escreva uma introdução de seção dessa forma!

Para a demonstação do \textsf{TFA} precisaríamos de todo um curso de Variáveis Complexas. 
Mas, como não é possível isso, considere que $\mathbb{C}[z]$ seja o conjunto dos 
polinômios complexos. 
Também vou considerear que você acreditará nos lemas seguintes:

\begin{lema}\label{lema:minimo}
  Seja $ f(z) \in \mathbb{C}[z] $. 
  Então, $ \vert f(z) \vert $ atinge um valor mínimo em algum ponto $ z_0 \in \mathbb{C} $
\end{lema}

\begin{lema}\label{lema:naoMinimo}
  Suponha que $ f(z) \in \mathbb{C}[z] $, sendo $ f(z) $ não constante. 
  Se $ f(z_0) \neq 0 $, então $ \vert f(z_0) \vert $ não é o valor mínimo de 
  $ \vert f(z) \vert $. 
\end{lema}

\begin{teorema}[Teorema Fundamental da Álgebra]
  Se $ f(z) \in \mathbb{C}[z] $, com $ f(z) $ não constante; então $ f(z) $ 
  possui, pelo menos, uma raíz complexa.
\end{teorema}

\begin{proof}[Demonstração.]
  Considere $ f(z) $ um polinômio complexo não constante. 
  Pelo Lema~\ref{lema:minimo}, $ \vert f(z) \vert $ atinge um valor mínimo em 
  algum ponto $ z_0 \in \mathbb{C} $. 
  Então, pelo Lema~\ref{lema:naoMinimo}, segue-se que $ f(z_0) = 0 $; pois, 
  caso contrário, não seria o valor mínimo. 
  Portanto, $ f(z) $ possui uma raíz complexa. 
\end{proof}

Quantos anos você possui?
O que tens feito da vida até aqui? 
Pois saiba que Gauss demonstrou esse teorema\footnote{
  não foi a demonstração acima, mas uma outra com as ferramentas disponíveis à época
} aos 18 anos de idade, em sua tese de doutorado! 

\begin{figure}[!htbp]
  \centering
  \includegraphics[width=0.3\linewidth]{leGauss}
  \caption{Gauss dando Legal}
  \label{fig:leGauss}
\end{figure}

Para mais detalhes, veja \textcite{TFA}
