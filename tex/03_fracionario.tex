% Sec 3: Cálculo Fracionário

\section{Cálculo Fracionário} % (fold)
\label{sec:fracionario}

\subsection{Derivada Fracionária} %-----------------------------------------

Você sabe que se $ f \colon X \to \mathbb{R} $ e $ a \in X \cap X^{\prime} $, 
a \textbf{derivada} da função $f$ no ponto $a$ é o limite:

\[
  f^{\prime}(a) = \lim_{h \to 0} \frac{f(a + h) - f(a)}{h}
\]

Usamos a notação $f^{n}(a)$ para a $n$-ésima derivada da função $f$. 
Quando existe $f^{n}(x)$ para todo $x\in I$, sendo $I$ um intervalo, dizemos que 
a função $f\colon I \to \mathbb{R}$ é \textit{n-vezes derivável no intervalo} $I$. 

A Tabela~\ref{tab:derivadas} lembra-nos de algumas derivadas notáveis

\begin{table}[!htbp]
  \centering
  \caption{Tabela de Derivadas}
  \label{tab:derivadas}
  \begin{tblr}{cc}
    \hline[1.6pt]
      $ f(x) $       &   $ f^\prime(x) $ \\
    \hline[0.8pt]    
      $ e^x $        &   $ e^x $ \\
      $ a^x $        &   $ a^x \ln{(a)} $ \\
      $ \arcsen{x} $ &   $ \frac{ 1 }{ \sqrt{1 - x^2} }\,\mathrm{d}x $ \\
    \hline[1.6pt]    
  \end{tblr}
\end{table}

O \textsf{\textbf{Cálculo Fracionário}} é uma área da Matemática que foi idealizada 
quando tentou-se generalizar a ideia da derivada para uma ordem \textit{arbitrária}, 
não apenas inteira. 

Você já imaginou uma derivada como $f^{1/2}(x)$? 
Ou, mais surpreendente: $f^{\sqrt{\pi}}(x)$??
Ou, absurda e mais surpreendente: $f^{2 + 3i}(z)$???

Pois é \ldots\ muitas definições surgiram ao longo da história, mas vou exibir 
uma formulação de Riemann-Liouville. 

\begin{definicao}[Derivada de Riemann-Liouville em intervalo finito]
  Uma das derivadas, em um intervalo fnito, de Riemann-Liouville, 
  $ D_{a^{+}}^{\alpha} f $, de ordem $ \alpha \in \mathbb{C} $ com 
  $\textrm{Re}(\alpha) \geq 0$ e $ \alpha \not \in \mathbb{N}$, é definida por:
  \begin{equation}
    \left(D_{a^{+}}^{\alpha}f\right)(x) = 
    \frac{1}{\Gamma(n - \alpha)}
    \left(
      \frac{\mathrm{d}}{\mathrm{d}x}
    \right)^{n} 
    \int_{a}^{x} 
    \frac
    { 
      f(t)\, \mathrm{d}t
    }
    {
      (x - t)^{\alpha - n + 1}
    }
  \end{equation}
com $ n = \left[\textrm{Re}(\alpha)\right] + 1 $ e $ x > a $, sendo 
$ \left[\textrm{Re}(\alpha)\right] $ é a parte inteira de \textrm{Re}(\alpha) e 
\[
  \Gamma(z) = \int_{0}^{\infty} t^{z - 1} e^{-t}\, \mathrm{d}t,
\]
com $\textrm{Re}(z) > 0$.
\end{definicao}

Sei que é complicado!
Mas, pior, é uma análise morfossintática, não?  

\subsection{Funções de Mittag-Leffler}

Uma função, denotada por $E_{\alpha}(z)$, possui papel relevante nos estudos 
do Cálculo Fracionário. 
Isto porque, podemos interpretá-la como uma generalização da função exponencial. 

\begin{definicao}[Função de Mittag-Leffler]
  A função de Mittag-Leffler, $E_{\alpha}(z)$, é uma função complexa que depende 
  de um parâmetro complexo $\alpha$, sendo $ \textrm{Re}(z) > 0 $, é dada pela 
  série de potências:
  \[
    E_{\alpha}(z) = \sum_{k = 0}^{\infty} \frac{z^k}{\Gamma(\alpha k + 1)}
  \]
\end{definicao}

Note que, se você demonstrar a igualdade $\Gamma(k + 1) = k!$, você perceberá 
a ideia da generalização da exponencial; pois, para $ \alpha = 1 $, temos:

\begin{align*}
  E_{1}(z) &= \sum_{k = 0}^{\infty} \frac{z^k}{\Gamma(k + 1)} \\
           &= \sum_{k = 0}^{\infty} \frac{z^k}{k!} \\
           &= e^{z}
\end{align*}

Se deseja conhecer mais sobre o Cálculo Fracionário, veja \textcite{fracionario}.
